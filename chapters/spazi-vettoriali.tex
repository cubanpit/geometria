\chapter{Spazi vettoriali} \label{ch:spazi-vettoriali}
\section{Proprietà principali} \label{sec:proprieta-spazi-vettoriali}
\begin{definizione} \label{d:spazio-vettoriale}
	Dato un campo $K$, un insieme $V$ non vuoto e due operazioni interne $+\colon V\times V\to V$ e $\cdot\colon K\times V\to V$, la terna $(V,+,\cdot)$ si definisce \emph{spazio vettoriale} sul campo $K$ se sono soddisfatte le seguenti proprietà:
	\begin{itemize}
		\item $(V,+)$ è un gruppo abeliano;
		\item $1_K x=x$ per ogni $x\in V$;
		\item la proprietà associativa, ossia se $\forall\lambda,\mu\in K$ e $\forall x\in V$, si ha $\lambda(\mu x)=(\lambda\mu) x$;
		\item la proprietà distributiva, ossia se $\forall\lambda,\mu\in K$ e $\forall x,y\in V$, si ha $(\lambda+\mu)x=\lambda  x+\mu x$ e $\lambda(x+y)=\lambda  x+\lambda  y$.
	\end{itemize}
\end{definizione}
Gli elementi di $V$ si chiamano \emph{vettori} mentre quelli di $K$ \emph{scalari}.
L'elemento neutro della somma, che per le proprietà note dei gruppi esiste ed è unico, sarà indicato con $0$, oppure $0_V$ in caso di ambiguità.
Lo zero e l'unità del campo $K$ seguono la convenzione già usata per la quale saranno indicati con $0$ e $1$, o anche $0_K$ e $1_K$; il fatto che $0$ indichi sia lo zero di $K$ che quello di $V$ sarà spesso chiaro dal contesto.
\paragraph{Esempi}
\begin{itemize}
	\item $(\R^n,+,\cdot)$, l'insieme delle $n$-uple ordinate di numeri reali, è uno spazio vettoriale su $\R$, infatti $\forall\lambda\in\R$ si ha, rappresentando i vettori come colonne,
		\begin{equation*}
			\lambda\cdot
			\begin{pmatrix}
				x_1\\\vdots\\x_n
			\end{pmatrix}
			=
			\begin{pmatrix}
				\lambda x_1\\\vdots\\\lambda x_n
			\end{pmatrix}
		\end{equation*}
		eccetera definendo somma e prodotto per scalare componente per componente.
	\item L'anello dei polinomi $\R[x]$ è uno spazio vettoriale su $\R$ con l'addizione e il prodotto per un numero reale, dove moltiplicare un polinomio per $\lambda\in\R$ equivale a moltiplicare per tale scalare tutti i suoi termini.
	\item L'insieme delle funzioni (qualunque) definite da un insieme $X\ne\emptyset$ e a valori reali forma uno spazio vettoriale, con le operazioni di addizione e prodotto per numero reale ``puntuali'', ossia $(f+g)(x)=f(x)+g(x)$ e $(\lambda f)(x)=\lambda f(x)$.
	\item Ogni campo può essere visto come spazio vettoriale su se stesso: ad esempio $(\R,+,\cdot)$ è uno spazio vettoriale su $\R$, e $(\C,+,\cdot)$ è uno spazio vettoriale su $\C$ ma anche su $\R$ se lo consideriamo come l'insieme delle coppie $(a,b)=a+ib$ con $a,b\in\R$.
\end{itemize}

Elenchiamo ora una serie di proprietà di base sugli spazi vettoriali, in cui assumiamo $V$ come spazio vettoriale su un campo $K$.
\begin{proprieta} \label{p:annullamento-vettore}
	Per ogni vettore $x\in V$, $0_K x=0_V$.
\end{proprieta}
\begin{proof}
	Lo $0_K$ si può sempre scrivere come somma di $0_K$ con se stesso, quindi $0_K x=(0_K+0_K)x=0_K x+0_K x$.
	Poiché $V$ è abeliano, sommando l'inverso di $0_K  x$ ai due membri si ottiene $0_K x=0_V$
\end{proof}
\begin{proprieta} \label{p:scalare-opposto}
	Per ogni scalare $a\in K$ e $\forall x\in V$, $-(ax)=(-a)x$.
\end{proprieta}
\begin{proof}
	Per la proprietà precedente si ha $0_V=0_K x$, e lo zero scalare si scrive come somma degli inversi $a+(-a)$, quindi $0_V=[a+(-a)]x=ax+(-a)x$, che significa che $(-a)x$ è il vettore inverso di $ax$ rispetto alla somma, ossia $(-a)x=-(ax)$.
\end{proof}
\begin{proprieta} \label{p:scalare-per-vettore-nullo}
	Per ogni $a\in K$, $a0_V=0_V$.
\end{proprieta}
\begin{proof}
	Si ha che $a0_V=a(0_V+0_V)=a0_V+a0_V$, e come per la proprietà \ref{p:annullamento-vettore} poiché $V$ è abeliano si somma ai due membri dell'uguaglianza l'inverso di $a0_V$, ottenendo $a0_V=0_V$.
\end{proof}
\begin{proprieta} \label{p:annullamento-prodotto-scalare-vettore}
	Se $ax=0_V$ per $a\in K$ e $x\in V$, allora $a=0_K$ o $x=0_V$.
\end{proprieta}
\begin{proof}
	Se $a=0_K$ è ovvia, se invece $a\neq 0_K$ allora esiste il suo inverso, $a^{-1}\in K$, rispetto al prodotto in $K$ (cioè tale che $aa^{-1}=1_K$).
	Quindi $0_V=a^{-1}0_V$, e poiché per per ipotesi $ax=0$ segue che $0_V=a^{-1}(ax)=(aa^{-1})x=1_K x=x$, perciò $x=0_V$.
\end{proof}
\begin{proprieta} \label{p:cancellazione-vettore}
	Per ogni $a,b\in K$ e per ogni $x\in V$, se $ax=bx$ allora $a=b$ oppure $x=0_V$.
\end{proprieta}
\begin{proof}
	Se vale che $ax=bx$, allora aggiungendo l'inverso di $bx$ per la somma si ottiene $ax-(bx)=0_V$.
	Inoltre per la proprietà distributiva questo è uguale ad $ax+(-b)x=(a-b)x=0_V$.
	Per la proprietà \ref{p:annullamento-prodotto-scalare-vettore}, infine, $a+(-b)=0_K$ oppure $x=0_V$.
	Sommando $b$ alla prima delle due risulta $a=b$ o $x=0_V$.
\end{proof}
\begin{proprieta} \label{p:cancellazione-scalare}
	Per ogni scalare $\lambda\in K$ e $\forall x,y\in V$, se $\lambda x=\lambda y$ allora $\lambda=0_K$ o $x=y$.
\end{proprieta}
\begin{proof}
	Da $\lambda x=\lambda y$ risulta $\lambda x+\big(-(\lambda y)\big)=\lambda x+\lambda(-y)=0_V$.
	Per la proprietà distributiva equivale a $\lambda\big(x+(-y)\big)=0_V$, da cui sempre per la \ref{p:annullamento-prodotto-scalare-vettore} $\lambda=0_K$ oppure $  x+(-y)=0_V$, da cui sommando $y$ ai due membri risulta $\lambda=0_K$ oppure $x=y$.
\end{proof}

\section{Sottospazi vettoriali} \label{ref:sottospazi-vettoriali}
\begin{definizione} \label{d:sottospazio-vettoriale}
	Sia $V$ uno spazio vettoriale sul campo $K$.
	Un suo sottoinsieme $W\subseteq V$ non vuoto si dice \emph{sottospazio} vettoriale se $(W,+,\cdot)$, con le operazioni indotte da $V$, è a sua volta uno spazio vettoriale.
\end{definizione}
In altre parole un sottospazio (ometteremo spesso l'attributo ``vettoriale'' per brevità) è un sottoinsieme che risulta chiuso rispetto alle due operazioni dello spazio vettoriale che lo contiene.
Per verificare che un insieme $W$ sia un sottospazio bisogna dunque provare che le combinazioni lineari di elementi di $W$ siano ancora in $W$: una condizione necessaria facile da verificare è che $W$ deve contenere lo $0_V$.

Ogni spazio vettoriale $V$ contiene sempre due spazi vettoriali, che sono banalmente $\{0_V\}$ e $V$ stesso.
Vediamone altri esempi.
\begin{itemize}
	\item Preso lo spazio vettoriale $\R^n$, l'insieme
		\begin{equation*}
			N=\left\{
			\begin{pmatrix}
				x_1\\\vdots\\x_{n-1}\\0
			\end{pmatrix}
			\colon x_1,\dots,x_{n-1}\in\R\right\}
		\end{equation*}
		è un sottospazio vettoriale, perché ognuna delle due operazioni dà sempre come risultato un vettore con l'$n$-esima componente nulla.
	\item Dato $\R[x]$, l'insieme dei polinomi di grado non maggiore di $n$, indicato con $\R_n[x]=\{p(x)=a_0+a_1x+\dots+a_nx^n\colon a_0,a_1,\dots,a_n\in\R\}$, formano un sottospazio vettoriale di $\R[x]$.
		Infatti la somma di due poliniomi di grado massimo $n$ è ancora un polinomio di grado massimo $n$, mentre moltiplicando un polinomio per uno scalare non nullo si moltiplicano i coefficienti di ogni termine per tale scalare, quindi il grado rimane immutato.
		Moltiplicando per zero si ottiene invece un polinomio nullo, che ha ancora ovviamente grado minore di $n$.
		Lo stesso vale per $\C_n[x]\leq \C[x]$.
	\item L'insieme $\cont{}(\R)$ delle funzioni definite da $\R$ a $\R$ e continue è un sottospazio vettoriale dello spazio delle funzioni $f\colon\R\to\R$.
		Infatti sommando due funzioni continue si ottiene una funzione continua, e ovviamente anche moltiplicando una funzione continua per uno scalare.
\end{itemize}

\begin{teorema} \label{t:intersezione-sottospazi}
	Sia $V$ uno spazio vettoriale su un campo $K$ e sia $\{W_i\}_{i\in I}$ un insieme di sottospazi vettoriali di $V$.
	Allora la loro intersezione	$\bigcap_{i\in I}W_i$ è ancora un sottospazio vettoriale di $V$.
\end{teorema}
\begin{proof}
	Siano $w_1,w_2\in\bigcap_{i\in I}W_i$.
	Allora $\forall i\in I$, $w_1$ e $w_2$ appartengono a $W_i$ (appartengono a tutti i sottospazi).
	Poiché i $W_i$ sono sottospazi vettoriali, allora accade sempre che $\forall i\in I$, $w_1+w_2\in W_i$, quindi appartengono anche a $\bigcap_{i\in I}$.
	Un ragionamento analogo si effettua per il prodotto per scalare.
	Quindi $\bigcap_{i\in I}$ è un sottospazio vettoriale di $V$.
\end{proof}
Il teorema non vale se al posto dell'intersezione si effettua l'unione dei $W_i$: ad esempio le due rette $x=0$ e $y=x$, rappresentate in forma vettoriale come $\big\{\big(\begin{smallmatrix} x\\0 \end{smallmatrix}\big)\colon x\in\R\big\}$ e $\big\{\big(\begin{smallmatrix} x\\x \end{smallmatrix}\big)\colon x\in\R\big\}$, sono banalmente due sottospazi vettoriali di $\R^2$.
Prendendo però un elemento del primo e uno del secondo, $\big(\begin{smallmatrix} 1\\0 \end{smallmatrix}\big)$ e $\big(\begin{smallmatrix} 1\\1 \end{smallmatrix}\big)$, sommandoli si ottiene $\big(\begin{smallmatrix} 2\\1 \end{smallmatrix}\big)$ che non appartiene all'unione dei due sottospazi.

\begin{definizione} \label{d:sottospazio-generato}
	Siano $V$ uno spazio vettoriale su $K$ e $S\subseteq V$ un insieme non vuoto.
	Si dice \emph{sottospazio generato} di $V$, e si indica con $\gen{S}$, un sottospazio vettoriale che soddisfa le seguenti due proprietà:
	\begin{itemize}
		\item $S\subseteq\gen{S}$;
		\item se $W\leq V$ tale che $S\subseteq W$, allora $\gen{S}\leq W$.
	\end{itemize}
\end{definizione}
\begin{teorema} \label{t:unicità-span}
	Siano $V$ uno spazio vettoriale su $K$ e $S\subseteq V$ un insieme non vuoto, esiste sempre $\gen{S}$ ed è unico.
\end{teorema}
\begin{proof}
	Mostriamo l'esistenza: costruiamo l'insieme $\gen{S}=\bigcap_{i\in I}Z_i$ dove $\{Z_i\}_{i\in I}$ sono tutti sottospazi vettoriali di $V$ che includono $S$; ogni $Z_i$ non è vuoto perché include $S$.
	Sicuramente $\gen{S}$ è, a sua volta, un sottospazio di $V$ per il teorema \ref{t:intersezione-sottospazi}.
	$S$ è contenuto in ogni $Z_i$, quindi è incluso anche in $\gen{S}=\bigcap_{i\in I}Z_i$.
	Inoltre, sia $W$ un sottospazio di $V$ tale che $S\subseteq W$: allora $S\subseteq\gen{S}\subseteq W$.
	Poiché $\gen{S}$ è un sottospazio vettoriale, per ogni $x,y\in S$ e $\lambda,\mu\in K$ ($x$ e $y$ appartengono a $\gen{S}$ e a $W$), mentre $\lambda x+\mu y\in\gen{S}$, quindi $\gen{S}\leq W$.
	Tale $\gen{S}$ rispetta dunque la definizione \ref{d:sottospazio-generato}.

	Vediamo ora l'unicità.
	Se $Z_1,Z_2$ sono due sottospazi vettoriali di $V$ che soddisfano la definizione \ref{d:sottospazio-generato}, allora $S\subseteq Z_1$ e $S\subseteq Z_2$.
	Se poi per un altro sottospazio vettoriale $W$ si ha $S\subseteq W$, allora sempre dalla definizione si deve avere $Z_1\le W$ e analogamente $Z_2\le W$.
	Ma anche $Z_2$ è un sottospazio di $V$ e $S\subseteq Z_2$, dunque $Z_1\le Z_2$, e allo stesso modo $Z_2\le Z_1$, quindi $Z_1=Z_2$.
\end{proof}

Definiamo ora la somma di sottospazi come l'insieme $U+W=\{u+w\colon u\in U,w\in W\}$: esso è un sottospazio vettoriale, infatti
\begin{equation*}
	\begin{gathered}
		(u_1+w_1)+(u_2+w_2)=(u_1+u_2)+(w_1+w_2)\in U+W\\
		\lambda(u+w)=\lambda u+\lambda w\in U+W.
	\end{gathered}
\end{equation*}
Dimostriamo inoltre che $U+W$ è lo spazio generato dall'unione dei due sottospazi, seguendo la definizione \ref{d:sottospazio-generato}.
\begin{teorema}
	Siano $U,W$ sottospazi vettoriali di $V$ su un campo $K$. Allora $\gen{U\cup W}\equiv U+W$.
\end{teorema}
\begin{proof}
	Ogni $u\in U$ si può scrivere come $u+0_W=u+0_V$ che quindi appartiene a $U+W$, quindi $U\subseteq U+W$ e analogamente $W\subseteq U+W$, quindi $U\cup W\subseteq U+W$.
	Consideriamo un sottospazio vettoriale $T$ di $V$ che includa $U\cup W$: ogni elemento $u+w$ appartiene anche a $T$ per qualunque $u$ e $w$, ma allora $U+W$ è un sottoinsieme di $T$ oltre che uno spazio vettoriale, e ciò lo rende un sottospazio vettoriale di $T$.
	Abbiamo allora dimostrato che $U+W$ soddisfa la definizione \ref{d:sottospazio-vettoriale}, perciò $U+W=\gen{U\cup W}$.
\end{proof}

\section{Sistemi di generatori} \label{sec:sistemi-generatori}
Sia $V$ uno spazio vettoriale su $K$, e $S\subseteq V$ un insieme non vuoto.
Le combinazioni lineari (sempre finite!) di elementi di $S$ sono definite come
\begin{equation*}
	\sum_{i=1}^n\lambda_i  s_i=\lambda_1  s_1+\lambda_2  s_2+\dots+\lambda_n  s_n,
\end{equation*}
con $\lambda_i\in K$, $  s_i\in S$ e $n\in\N$.

\begin{teorema}
	Sia $V$ uno spazio vettoriale su $K$, e $S\subseteq V$ non vuoto. Allora
	\begin{equation*}
		\gen{S}=\left\{\sum_{i=1}^n\lambda_i  s_i\colon \lambda_i\in K,   s_i\in S, i=1,2,\dots,n, n\in\N\right\}.
	\end{equation*}
\end{teorema}
\begin{proof}
	Questo particolare $\gen{S}$ deve soddisfare la definizione \ref{d:sottospazio-generato}:
	\begin{itemize}
		\item i $  s_i$ appartengono a $S$, e possiamo esprimerli come $  s_i=1_K  s_i$ quindi $S\subseteq\gen{S}$;
		\item se $W\leq V$ e $S\subseteq W$, allora dato che $\gen{S}\supseteq S$ se prendiamo una combinazione lineare di due elementi di $S$, lo è anche di elementi di $W$, e poiché il risultato è sempre un elemento di $\gen{S}$ quest'ultimo è un sottospazio vettoriale di $W$.
	\end{itemize}
\end{proof}

\begin{definizione}
	Sia $V$ uno spazio vettoriale sul campo $K$ e sia $S\subseteq V$ un insieme non vuoto. $S$ è detto \emph{sistema di generatori} per $V$ se $\gen{S}=V$.
\end{definizione}
Con questa definizione possiamo studiare anziché l'intero spazio vettoriale $V$ solo un suo sottoinsieme.
\paragraph{Esempi}
\begin{itemize}
	\item Come già detto, i vettori di $\R^n$ sono definiti dalle loro coordinate, quindi possono essere scritti come combinazioni lineari di questi elementi: allora
		\begin{equation*}
			\R^n=\gen{
				\begin{pmatrix}
					1\\0\\\vdots\\0
				\end{pmatrix},
				\begin{pmatrix}
					0\\1\\\vdots\\0
				\end{pmatrix},\dots,
				\begin{pmatrix}
					0\\0\\\vdots\\1
				\end{pmatrix}
			}
		\end{equation*}
		I vettori dello spazio generatore sono a tutti gli effetti dei versori di $\R^n$, in questo esempio sono i versori allineati con gli assi cartesiani.
	\item $\R[x]$ è generato da $\{1,x,x^2,\dots,x^n,\dots\}$; questo insieme è infinito, perché non esiste un polinomio ``di grado massimo''.
		Ogni $  x\in\R[x]$ è determinato da una combinazione lineare di questi componenti, in modo univoco.
	\item In $\R^2$ si può individuare il sistema di generatori $\gen{\big(\begin{smallmatrix} 0\\1 \end{smallmatrix}\big),\big(\begin{smallmatrix} 1\\0 \end{smallmatrix}\big),\big(\begin{smallmatrix} 1\\1 \end{smallmatrix}\big)}$.
		Con questo insieme però si può scrivere l'elemento $\big(\begin{smallmatrix} 2\\2 \end{smallmatrix}\big)$ in due modi diversi, ossia come $2\big(\begin{smallmatrix} 0\\1 \end{smallmatrix}\big)+2\big(\begin{smallmatrix} 1\\0 \end{smallmatrix}\big)+0\big(\begin{smallmatrix} 1\\1 \end{smallmatrix}\big)$ ma anche come $0\big(\begin{smallmatrix} 0\\1 \end{smallmatrix}\big)+0\big(\begin{smallmatrix} 1\\0 \end{smallmatrix}\big)+2\big(\begin{smallmatrix} 1\\1 \end{smallmatrix}\big)$.
		Questo sistema di generatori quindi non permette di scrivere in maniera univoca i vettori di $\R^2$.
\end{itemize}

\begin{definizione} \label{d:dipendenza-lineare}
	Sia $V$ uno spazio vettoriale su $K$, e $\{  v_i\}_{i\in I}\subseteq V$.
	Si dice che l'insieme $\{v_i\}$ è \emph{linearmente dipendente} se esiste $I_0\subseteq I$, di cardinalità $n$ finita, e un insieme di scalari $\{\lambda_1,\lambda_2,\dots,\lambda_n\}\in K\setminus\{0_K\}$ tali per cui
	\begin{equation*}
		\sum_{i=1}^n\lambda_i  v_i=0_V,
	\end{equation*}
	dove $\{  v_i\}_{i=1}^n$ è una numerazione di $\{  v_i\}_{i\in I}$.
\end{definizione}
In parole povere, un insieme è linearmente dipendente se esiste almeno una combinazione lineare (con i coefficienti non tutti nulli) dei suoi componenti che dia lo zero dello spazio.
Per l'ultimo degli esempi precedenti l'insieme $\big\{\big(\begin{smallmatrix} 0\\1 \end{smallmatrix}\big),\big(\begin{smallmatrix} 1\\0 \end{smallmatrix}\big),\big(\begin{smallmatrix} 1\\1 \end{smallmatrix}\big)\big\}$ è linearmente dipendente.

Ovviamente, un sistema che non è linearmente dipendente si dice \emph{linearmente indipendente}, o anche \emph{libero}.

\begin{definizione} \label{d:indipendenza-lineare}
	Un insieme finito di vettori $\{  v_1,  v_2,\dots,  v_k\}$ si dice \emph{linearmente indipendente}, se in ogni combinazione lineare dei $k$ vettori che produce $0_V$ i coefficienti sono tutti nulli:
	\begin{equation*}
		\lambda_1  v_1+\lambda_2  v_2+\dots+\lambda_k  v_k=0_V\ \then\ \lambda_1=\lambda_2=\dots=\lambda_k=0_K.
	\end{equation*}
	Un insieme infinito di vettori $\{  v_i\}_{i\in I}$ (con $I$ quindi anche di cardinalità infinita) è linearmente indipendente se $\forall J\subseteq I$ di cardinalità finita $\{  v_j\}_{j\in J}$ è linearmente indipendente (cioè se lo è ogni suo sottoinsieme).
\end{definizione}
Alcuni esempi di insiemi linearmente indipendenti:
\begin{itemize}
	\item il sistema che genera $K_n[x]$, ossia $\{1,x,x^2,\dots,x^n\}$, è linearmente indipendente perché un polinomio è identicamente nullo se e solo se tutti i coefficienti dei vari termini sono nulli.
		Lo stesso vale per i polinomi di grado non limitato di $K[x]$, poiché la definizione è verificata da ``blocchi'' di termini.
	\item l'insieme $\{  v,  w,0_V,  z\}\subset V$ spazio vettoriale su $K$ non lo è, poiché $0_K  v+0_K  w+1_K0_V+0_K  z=0_V$ anche se uno dei coefficienti, $1_K$, non è nullo.
\end{itemize}
Il seguente teorema indica un modo più semplice di verificare questa definizione.
\begin{teorema}
	Un insieme di vettori $\{  v_i\}_{i\in i}\subset V$ è linearmente dipendente se e solo se almeno uno di essi è una combinazione lineare di un numero finito dei rimanenti.
\end{teorema}
\begin{proof}
	Sia dato l'insieme, linearmente dipendente, $\{  v_i\}_{0<i\leq k}$: esiste una combinazione lineare $\sum_{n=1}^k\lambda_n  v_n$ nulla senza che tutti i $\lambda_n$ siano nulli.
	Trascurando nella serie gli eventuali termini nulli, rimangono un numero finito di termini tali che ad esempio $\lambda_1  v_1=-\lambda_2  v_2-\dots-\lambda_n  v_n$.
	Poiché $\lambda_1$ non è nullo, esiste il suo inverso rispetto al rapporto, $(\lambda_1)^{-1}$, e moltiplicando la precedente equazione per questo risulta che il primo termine è $(\lambda_1)^{-1}(\lambda_1  v_1)=(\lambda_1^{-1}\lambda_1)  v_1=1_K  v_1=v_1$, allora
	\begin{equation*}
	  v_1=(\lambda_1^{-1})(-\lambda_2  v_2-\dots-\lambda_n  v_n),
	\end{equation*}
	che è quindi combinazione lineare degli altri vettori dell'insieme.

	Sia $  v^*\neq 0_V$ un vettore dell'insieme dato, combinazione lineare (in cui quindi i coefficienti non possono essere tutti nulli) di alcuni dei vettori rimanenti, quindi
	\begin{equation*}
		  v^*=\mu_1  v_1+\mu_2  v_2+\dots+\mu_r  v_r.
	\end{equation*}
	Portando tutto al primo termine risulta $  v^*-\mu_1  v_1-\mu_2  v_2-\dots-\mu_r  v_r=0_V$ sebbene non siano tutti nulli.
\end{proof}

\section{Basi e dimensioni} \label{sec:basi-dimensioni}
\begin{definizione} \label{d:base}
	Si chiama \emph{base} di uno spazio vettoriale $V$ ogni sistema $S$ linearmente indipendente che genera $S$.
\end{definizione}
	Un sistema di generatori esiste sempre per ogni spazio non vuoto: semmai si può prendere lo spazio stesso; non è sempre certa, però, l'esistenza di un insieme linearmente indipendente.
\begin{teorema}
	Sia $\{  e_i\}_{i\in I}$ un insieme di $V$.
	Esso è una base di $V$ se e solo se ogni elemento $  v\in V$ (non nullo) si può scrivere in modo univoco come combinazione lineare finita, a coefficienti non nulli, di elementi di $\{  e_i\}_{i\in I}$.
\end{teorema}
Gli elementi $  v$ non devono essere nulli, perché $0_V$ si può scrivere come combinazione lineare di \emph{qualunque} sistema di vettori; inoltre i coefficienti della combinazione non devono essere nulli, altrimenti si potrebbe affermare che $  v=a  e_1+b  e_2$ ma anche $  v=a  e_1+b  e_2+0_K  e_3+\dots+0_K  e_n+\dots$ quanto si vuole.
\begin{proof}
	Dimostriamo che la condizione è necessaria.
	Sia $\{  e_i\}_{i\in I}$ una base di $V$: allora genera tutto $V$.
	Preso un elemento $  v\in V$ non nullo, si può scrivere come la combinazione lineare finita
	\begin{equation} \label{eq:dim-base1}
		  v=\sum_{i\in I_0}\lambda_i  e_i,
	\end{equation}
	con $I_0\subset I$ di cardinalità finita.
	Dimostriamo che questa scrittura è unica: supponiamo che $\forall i\in I_0$ $\lambda_i\neq 0_K$ (eventuali termini nulli nella combinazione lineare si trascurano); allora esiste anche 
	\begin{equation} \label{eq:dim-base2}
		  v=\sum_{j\in J_0}\mu_j  e_j
	\end{equation}
	con $\mu_j\neq 0_K$ $\forall j\in J_0\subset I$ e di cardinalità finita.
	Sommando gli opposti della \eqref{eq:dim-base2} alla \eqref{eq:dim-base1} si ha
	\begin{equation*}
		\sum_{i\in I_0}\lambda_i  e_i+\sum_{j\in J_0}-\mu_j  e_j=0_V.
	\end{equation*}
	Sia $I_0\subseteq J_0$, e ipotizziamo che esista $j_0$ appartenente a $J_0$ ma non a $I_0$.
	Allora l'elemento $  e_{j_0}$, nella combinazione, è associato al coefficiente $-\mu_{j_0}\neq 0_K$ (quindi esiste il suo reciproco).
	Portando $-\mu_{j_0}  e_{j_0}$ al secondo membro dell'uguaglianza e moltiplicando per il reciproco di $\mu_{j_0}$ risulta
	\begin{equation*}
		\sum_{i\in I_0}(\lambda_i\mu_{j_0}^{-1})  e_i+\sum_{j\in J_0}(\mu_j\mu_{j_0}^{-1})  e_j=  e_{j_0},
	\end{equation*}
	cioè uno degli elementi di $\{  e_i\}$ è espresso come combinazione lineare degli altri, vale a dire che la base è linearmente dipendente, il che è assurdo perché è una base: quindi non può esistere un $j_0$ che appartiene a $J_0$ ma non a $I_0$.
	Ponendo $J_0\subseteq I_0$ si ottiene allo stesso modo un altra contraddizione.
	Allora non può che essere $I_0\equiv J_0$, ma ciò significa che
	\begin{equation*}
		\sum_{i\in I_0}(\lambda_i-\mu_i)  e_i=0_V,
	\end{equation*}
	cui segue che $\lambda_i=\mu_i$ $\forall i\in I_0$, cioè le \eqref{eq:dim-base1} e \eqref{eq:dim-base2} sono identiche e dunque la scrittura di $  v$ in termini della base è unica.

	Mostriamo ora che la condizione è anche sufficiente: innanzitutto, $\gen{\{  e_i\}_{i\in I}}=V$ perché per ipotesi possiamo scrivere ogni vettore di $V$ come combinazione lineare di elementi di questo insieme.
	Supponiamo che esista $I_0\subset I$ di cardinalità finita, per cui
	\begin{equation} \label{eq:dim-base3}
		\sum_{i\in I_0}\lambda_i  e_i=0_V,
	\end{equation}
e come prima che $\lambda_i\neq 0_K$ $\forall i\in I_0$.
	Considerando un $i^*\in I_0$, $\lambda_{i^*}$ non è nullo, quindi moltiplicando la \eqref{eq:dim-base3} per il suo inverso si trova
	\begin{equation*}
		\sum_{i\in I_0}(\lambda_{i^*}^{-1}\lambda_i)  e_i=0_V.
	\end{equation*}
	Isolando il termine in $i^*$ si ottiene poi che
	\begin{equation*}
		\sum_{i^*\neq i\in I_0} (\lambda_{i^*}^{-1}\lambda_i)  e_i=-  e_{i^*}
	\end{equation*}
	vale a dire che $  e_{i^*}$ si esprime come combinazione lineare di altri elementi dell'insieme.
	Questo elemento però non è unico, dato che il ragionamento vale per qualsiasi $i\in I_0$ preso volta per volta.
	Allora è assurdo che esista un tale $I_0$, cioè che per tali $i\in I_0$ la combinazione lineare sia nulla pur non avendo tutti i coefficienti nulli; dunque l'insieme è linearmente indipendente, e dato che genera $V$ è una sua base.
\end{proof}
In $\R^3$ ad esempio ogni punto è univocamente individuato da un vettore $  v$ che è esprimibile nelle sue (tre) coordinate come $  v=xe_1+ye_2+ze_3$, infatti l'insieme $\{e_1,e_2,e_3\}=\{(1,0,0),(0,1,0),(0,0,1)\}$ è una base di $\R^3$ (detta comunemente \emph{base canonica}).
I coefficienti di ogni termine formano le componenti del vettore.

\begin{definizione}
	Sia $V$ uno spazio vettoriale su $K$: esso si dice di dimensione finita se ammette un sistema finito di generatori, altrimenti si dice di dimensione infinita.
\end{definizione}
\begin{teorema} \label{t:esistenza-base}
	Sia $G\subset V$ un sistema di generatori di $V$ finito.
	Se $S$ è un sottoinsieme di $G$ linearmente indipendente, allora esiste una base $B$ di $V$ tale da comprendere il sistema $S$ e che sia inclusa in $G$ (cioè $S\subseteq B\subseteq G$).
\end{teorema}
\begin{proof}
	Si supponga che $V$ abbia dimensione finita: allora esiste un sistema di generatori $G$.
	Escludiamo il caso in cui $V\equiv\{0_V\}$ perché non esisterebbe nemmeno una base.
	Esiste $  e\in G$, non nullo, che ovviamente forma un sistema linearmente indipendente poiché $\lambda  e\neq 0_V$ $\forall\lambda\neq 0_K$.
	Inoltre, per come si è scelto $  e$, $S\equiv\{  e\}\subseteq G$.
	Quindi esiste sempre una base di $V$ per cui $\{  e\}\subseteq B\subseteq G$.

	Indichiamo con $S_n$ il fatto che nell'insieme linearmente indipendente $S$ ci siano $n$ vettori.
	Potrebbe essere che $\gen{S_n}\equiv V$, ma allora possiamo scegliere subito $S_n$ come base per $V$, e poiché $B=S_n\subseteq G$ il teorema è dimostrato.
	Sia allora $\gen{S_n}\lneqq V$: deve esistere $  x_{n+1}\in G$, ma $\notin\gen{S_n}$, perché altrimenti dato che $\gen{S}\supseteq G$ si avrebbe che $\gen{S_n}\equiv V$.
	Definiamo $S_{n+1}=S_n\cup\{  x_{n+1}\}$ (quindi l'insieme ha $n+1$ elementi), per cui sicuramente vale $S_n\subseteq S_{n+1}\subseteq G$. Questo $S_{n+1}$ è un insieme linearmente indipendente, altrimenti avremmo che $  x_{n+1}\in\gen{S_n}$.
	Se $\gen{S_{n+1}}=V$ il teorema è dimostrato, altrimenti procediamo aggiungendo un altro elemento di $G\setminus\gen{S_{n+1}}$.
	Iteriamo il processo per $n+2$, $n+3$ e così via, fino a quando si esauriscono gli elementi di $G$, ottenendo la relazione
	\begin{equation*}
		S_n\subseteq S_{n+1}\subseteq S_{n+2}\subseteq\dots\subseteq G.
	\end{equation*}
	La dimensione di $G$ è finita, quindi prima o poi gli elementi da aggiungere termineranno: esisterà $k\in\N$ per cui $\gen{S_{n+k}}=\gen{G}$, e anche in questo caso abbiamo trovato che $S_{n+k}$ è base di $V$.
\end{proof}
Sempre in $\R^3$, per esempio, una delle possibili basi è quella composta dai tre versori $\{\ii,\jj,\kk\}=\{(1,0,0),(0,1,0),(0,0,1)\}$, ma anche $\{  e_1,  e_2,  e_3\}=\{\ii,\jj,(\jj+\kk)\}$ è un'altra base.
In effetti ruotando $\ii$, $\jj$ e $\kk$ di un angolo qualsiasi si ottiene un altra base, e se ne ottengono ancora delle altre moltiplicando per degli scalari (anche differenti) i tre versori.
Quindi le basi di uno spazio vettoriali sono infinite; quello che non cambia è il numero di elementi di queste basi, che è sempre costante (in questo esempio, la base è sempre composta da tre vettori).
\begin{corollario}
	Ogni spazio vettoriale $V\neq\{0_V\}$ di dimensione finita ammette almeno una base.
\end{corollario}
\begin{teorema} \label{t:base-dimensione}
	Sia $V$ uno spazio vettoriale di dimensione finita, contenente una base di $n$ vettori.
	Allora:
	\begin{enumerate}
		\item ogni sistema linearmente indipendente $S$ di $n$ vettori è una base di $V$;
		\item ogni sistema $U$ di $m>n$ vettori è linearmente dipendente;
		\item ogni sistema $W$ di $m<n$ vettori non può generare $V$, cioè $\gen{W}\neq V$;
		\item ogni sistema $T$ di $n$ vettori per cui $\gen{T}=V$ è una base.
	\end{enumerate}
\end{teorema}
\begin{proof}
\begin{verbatim}

\end{verbatim}
	\begin{enumerate}
		\item Siano $\{  e_i\}_{i=1}^n$ una base di $V$, e $S=\{  f_i\}_{i=1}^n$ un insieme finito linearmente indipendente.
			Allora
			\begin{equation*}
				  f_1=\sum_{i=1}^n\lambda_i  e_i.
			\end{equation*}
			Poiché $S$ è linearmente indipendente, almeno un $\lambda_i$ non è nullo, quindi $  f_1\neq 0_V$, e riordinando i vettori nella combinazione possiamo supporre che sia $\lambda_1\neq 0_K$: in questo modo $\lambda_1  e_1\neq 0_V$.
			Portando quest'ultimo termine al secondo membro e moltiplicando per $\mu_1=\lambda_1^{-1}$ risulta con opportuni $\mu_i\in K$ che
			\begin{equation*}
				  e_1=\mu_1  f_1+\sum_{i=2}^n\mu_i  e_i.
			\end{equation*}
			Ogni vettore di $V$ è una combinazione lineare di elementi di $\{  e_i\}_{i=1}^n$, ma sostituendo $  e_1$ con l'espressione trovata sopra abbiamo $\forall  v\in V$
			\begin{equation*}
				  v=\sum_{i=1}^n\lambda_i  e_i=\lambda_1\bigg(\mu_1  f_1+\sum_{i=2}^n\mu_i  e_i\bigg)+\sum_{i=2}^n\lambda_i  e_i
			\end{equation*}
			che quindi può essere espresso anche come combinazione lineare di $\{  f_1,  e_2,\dots,  e_n\}$ anziché degli $\{  e_i\}_{i=1}^n$, quindi anche l'insieme $\{  f_1,  e_2,\dots,  e_n\}$ è un sistema di generatori di $V$.
			Dunque troviamo anche che
			\begin{equation*}
				  f_2=\sigma_1  f_1+\sum_{i=1}^n\sigma_i  e_i.
			\end{equation*}
			Almeno uno dei $\sigma_i$ non è nullo, altrimenti sarebbe che $  f_2=\sigma_1  f_1$ che contraddice l'indipendenza lineare degli $  f_i$.
			Supponendo $\sigma_2\neq 0_K$, si esplicita $  e_2$ moltiplicando per $\sigma_2^{-1}$, ottenendo
			\begin{equation*}
				  e_2=\rho_1  f_1+\rho_2  f_2+\sum_{i=3}^n\rho_i  e_i.
			\end{equation*}
			L'insieme $\{  f_1,  f_2,  e_3,\dots,  e_n\}$ è ancora un sistema di generatori di $V$.
			Si itera il procedimento ottenendo alla fine che $\{  f_1,  f_2,\dots,  f_n\}$ è ancora un sistema di generatori per $V$, e dunque ne è una base dato che è linearmente indipendente. 
		\item Sia $U$ con $m>n$ elementi linearmente indipendente, e si prenda $U'\subset U$ tale che abbia $n$ elementi (dunque $U\setminus U'\neq\emptyset$).
			Per il punto 1 $U'$ è una base di $V$, quindi i vettori di $U'$ generano anche quelli di $U\setminus U'$.
			Ciò contraddice l'indipendenza lineare di $U$, che deve essere quindi linearmente dipendente.
		\item Sia $W$ con $m<n$ elementi un sistema di generatori di $V$: allora deve esistere una base con al più $m$ vettori, tanti quanti ce ne sono in $W$.
			Per il punto precedente, la base $\{  e_i\}_{i\in I}$ (considerata nel punto 1) ha più vettori della base estratta da $W$, quindi sarebbe linearmente dipendente, che è assurdo.
			Allora $W$ non può essere un sistema di generatori di $V$.
		\item Se $\gen{T}=V$, $T$ deve avere almeno $n$ elementi per il punto 3; allora esiste $T'\subseteq T$ che è una base di $V$.
			Se $T'$ avesse meno di $n$ elementi, contraddirrebbe il punto 3 prima citato, quindi deve averne esattamente $n$, perciò $T\equiv T'$, e $T$ è linearmente indipendente.
			Poiché genera $V$, $T$ ne è anche una base.
	\end{enumerate}
\end{proof}
\begin{corollario}
	Sia $V$ uno spazio vettoriale di dimensione finita, contenente una base di $n$ vettori.
	Ogni altra base $V$ ha a sua volta esattamente $n$ vettori.
\end{corollario}
\begin{definizione} \label{d:dimensione}
	Dato uno spazio vettoriale $V\neq\{0_V\}$ su $K$ di dimensione finita, si dice \emph{dimensione} di $V$ su $K$ il numero di vettori di una sua base qualunque.
\end{definizione}
La dimensione di $V$ (su $K$) si indica con $\dim_KV$ o anche solo, se non ci sono ambiguità, con $\dim V$. Convenzionalmente, allo spazio contenente soltanto $\{0_V\}$ si assegna la dimensione 0.
\begin{teorema}
	Sia $V$ uno spazio vettoriale, e $W$ un suo sottospazio.
	Se $\dim V$ è finita, allora $\dim W\leq\dim V$.
\end{teorema}
\begin{proof}
	Nel caso banale in cui $W=\{0_V\}$, la sua dimensione è 0 quindi è ovviamente minore o uguale della dimensione di $V$, qualunque essa sia.
	Se invece $W$ non contiene soltanto il vettore nullo, una base qualunque di $W$ è un sistema linearmente indipendente anche in $V$.
	Siccome $W\neq\{0_V\}$, esiste un vettore $  w_1\in W$ non nullo.
	Se $\gen{  w_1}=W$ allora $\{  w_1\}$ è una base di $W$ da cui si ottiene subito la tesi: infatti $  w_1$ appartiene anche a $V$ che dunque ha almeno dimensione 1.
	Altrimenti $\gen{  w_1}\neq W$ ma comunque esiste un altro vettore di $W$, $  w_2\notin\gen{  w_1}$ (quindi appartenente a $W\setminus\gen{  w_1}$).
	Sia $S=\{  w_1,  w_2\}$: esso è un sistema linearmente indipendente per come abbiamo scelto $  w_2$.
	%Nella combinazione lineare $\lambda_1  w_1+\lambda_2  w_2=0_V$ non può essere $\lambda_1=0$ perché $\{  w_1\}$ è un sistema libero, mentre ugualmente $\lambda_2\neq 0$ perché\dots?
	Se $\gen{S}=W$, come nel caso precedente abbiamo dimostrato il teorema: per teorema precedente (\ref{t:base-dimensione}) $V$ ha due vettori linearmente indipendenti dunque deve essere $\dim V\geq 2=\dim W$.
	Se invece $\gen{S}\neq W$, si consideri un altro $  w_3\in W\setminus\gen{  w_1,  w_2}$ e si ripete il ragionamento compiuto in precedenza.
	Ogni volta si trova un nuovo insieme $S=\{  w_1,  w_2,\dots,  w_n\}$, ancora linearmente indipendente in $V$.
	Per il punto 3, sempre del teorema \ref{t:base-dimensione}, $\dim V\geq\abs{S}$, quindi $W$ ha dimensione finita, inoltre $n=\dim W=\abs{S}\leq\dim V$.
\end{proof}

\begin{teorema}\label{estensione-base}
	Sia $V$ uno spazio vettoriale di dimensione finita, $W$ un suo sottospazio e $B_W$ una base di $W$.
	Allora tale base si può estendere per formare una base di $V$, cioè $\exists B_V\colon B_W\subseteq B_V$.
\end{teorema}
\begin{proof}
	$B_W$ è linearmente indipendente in $W$ in quanto è una base, e lo è quindi anche in $V$ (quindi $\dim V\geq\abs{B_W}$), che deve avere un sistema finito $G$ di generatori.
	Allora $B_W\cup G$ è ancora un sistema di generatori per $V$, ed è anche finito.
	Da questo insieme, che per generare $V$ deve essere tale che $\dim V\leq\abs{B_W\cup G}$, possiamo estrarre $\dim V$ elementi linearmente indipendenti.
	Poiché $B_W\subseteq B_W\cup G$ è già linearmente indipendente, esiste una base $B_V$ di $V$ contenente $B_W$ (e contenuta in $B_W\cup G$).
\end{proof}
Ad esempio, $\R$ è un sottospazio di $\R^3$: prendendo una base $\{  v\}$ del primo, si ottiene una base del secondo semplicemente aggiungendo due vettori (distinti) perpendicolari a $  v$.

\begin{definizione} \label{d:insieme-linearmente-indipentente}
	Un insieme $\{V_i\}_{i\in I}$ di sottospazi vettoriali di $V$ si dice linearmente indipendente se comunque si scelga un vettore $x_i\neq 0$ per ciascun $V_i$, l'insieme $\{x_i\}_{i\in I}$ è linearmente indipendente.
\end{definizione}
\begin{teorema}
	Un insieme $\{V_i\}_{i\in I}$ di sottospazi vettoriali di $V$ è linearmente indipendente se e solo se $\forall k\in I$ si ha che l'intersezione tra $V_k$ e lo spazio generato dai restanti $V_i$ contiene soltanto lo zero, cioè\footnote{Ricordiamo che la somma di più spazi vettoriali è lo spazio generato dalla loro unione, ossia $\sum_i V_i=\gen{\bigcup_i V_i}$.}
	\begin{equation*}
		V_k\cap\sum_{j\in I\setminus\{k\}}{V_j}=\{0\}.
	\end{equation*}
\end{teorema}
\begin{proof}
	Se i sottospazi $V_i$ sono linearmente indipendenti, e per assurdo $V_k\cap\sum_{j\in I\setminus\{k\}}{V_j}\neq\{0\}$ per qualche $k\in I$, allora esisterebbe un elemento $v_k\in V_k$ non nullo che è combinazione lineare di elementi dei restanti sottospazi, cioè $v_k=x_{i_1}+\dots+x_{i_r}$ con $\{i_1,\dots,i_r\}\subseteq I\setminus\{k\}$.
	Ma allora l'insieme $\{V_i\}_{i\in I}$ dei sottospazi non sarebbe linearmente indipendente, contraddicendo l'ipotesi, quindi l'uguaglianza deve essere vera.

	Viceversa, se prendessimo una combinazione lineare nulla di elementi uno da ciascun sottospazio
	\begin{equation}
		a_{i_1}v_{i_1}+\dots+a_{i_k}v_{i_k}=0
		\label{eqdim:sottospazi-linearmente-indipendenti}
	\end{equation}
	con $v_{i_j}\in V_{i_j}$ appartenenti tutti a sottospazi distinti, e ci fosse $a_{i^*}\ne 0$, allora potremmo scrivere
	\begin{equation}
		v_{i^*}=\sum_{j\in I\setminus\{i^*\}}c_jv_j
	\end{equation}
	ma allora $v_{i^*}\in V_{i^*}$ e anche $v_{i^*}\in\sum_{j\in I\setminus\{i^*\}}V_j$, ossia esiste un $i^*\in I$ tale per cui
	\begin{equation*}
		V_{i^*}\cap\sum_{j\in I\setminus\{i^*\}}V_j\ne\{0\}
	\end{equation*}
	che contraddice l'ipotesi.
	Dunque nella \eqref{eqdim:sottospazi-linearmente-indipendenti} si ha $a_j=0$ per ogni $j\in I$, cioè i sottospazi sono linearmente indipendenti.
\end{proof}

\begin{teorema} \label{t:unione-sottospazi-linearmente-indipendente}
	Sia $\{V_i\}_{i\in I}$ un insieme linearmente indipendente di sottospazi vettoriali di $V$.
	Se $\forall i\in I$ il sistema di vettori $S_i\subset V_i$ è linearmente indipendente, allora $\bigcup_{i\in I}S_i$ è linearmente indipendente in $V$.
\end{teorema}
\begin{proof}
	Consideriamo un sistema linearmente indipendente di vettori $S_i\subset V_i$ e ipotizziamo per assurdo che l'unione non sia linearmente indipendente.
	Questo implica che, considerando gli elementi $y^k_i\in S_i$ con $i,k\in I_0$ in cui $I_0\in I$ ha cardinalità finita, 
	\begin{equation*}
		\sum_{1\leq k \leq n_i} \lambda^k_i y_k^i = 0,
	\end{equation*}
	per ogni $\lambda^k_i$ meno un $\lambda^k_{\tilde{i}}$, corrispondente a un $\tilde{k}$.
	Posto $  x_i = \sum\limits_{1\leq k \leq n_i} \lambda^k_i x_k^i$ sappiamo che $\sum_{i\in I_0}  x_i = 0$ e che un elemento di questa sommatoria non è nullo, ma per ipotesi $  x_i\subset S_i$ che è una combinazione di vettori linearmente indipendente, allora deve essere $  x_{\tilde{i}} = 0$.
	L'unione $S$ dei vari $S_i$ è allora linearmente indipendente. 
\end{proof}

\begin{definizione} \label{d:somma-diretta}
	Uno spazio vettoriale $V$ si dice \emph{somma diretta} di un insieme di sottospazi vettoriali $\{V_i\}_{i\in I}$ se $\{V_i\}_{i\in I}$ è linearmente indipendente e se $\sum_{i\in I}V_i\equiv V$.
\end{definizione}
Per indicare che $V$ è composto dalla somma diretta degli spazi $V_i$ si usa la scrittura $V=\bigoplus\limits_{i\in I}V_i$.
\paragraph{Esempi}
\begin{itemize}
	\item Lo spazio $\R[x]$ è generato dall'insieme $\{1,x,x^2,\dots\}$.
		Posto $V_j=\gen{x^j}$, con $j\in\N_0$, si ha che
		\begin{equation*}
			\R[x]=\bigoplus_{j\in\N_0}V^j=\bigoplus_{j\in\N_0}\gen{x^j}.
		\end{equation*}
	\item In $\R^3$, siano $V_1=\gen{\begin{psmallmatrix}1\\0\\0\end{psmallmatrix},\begin{psmallmatrix}0\\1\\0\end{psmallmatrix}}$ e $V_2=\gen{\begin{psmallmatrix}0\\1\\0\end{psmallmatrix},\begin{psmallmatrix}0\\0\\1\end{psmallmatrix}}$.
		Certamente $V_1+V_2=\R^3$, ma la somma non è diretta poiché la loro intersezione è l'asse $y$.
\end{itemize}

\section{Spazi quoziente} \label{sec:spazi_quoziente}
Sia $W$ un sottospazio vettoriale di $V$.
Definiamo la relazione $  x\sim   y$, con $x,y\in V$, se $  x-  y\in W$: essa è di equivalenza, perché soddisfa le tre proprietà della definizione \ref{d:relazione-equivalenza}.
Infatti $  x-  x=0_V$ che certamente appartiene a $W$; se $  x-  y=  w\in W$ esiste, allora deve esistere il suo opposto in $W$ (ed esiste, dato che $W$ è uno spazio vettoriale) che è $-  w=  y-  x\in W$; infine se $  x-  y=  w_1$ e $  y-  z=  w_2$ sono vettori di $W$, allora sommandoli si ottiene $  w_1-  w_2=  x-  y+  y-  z=  x-  z$ che poiché $W$ è uno spazio vettoriale, vi appartiene.
Prendiamo un elemento $  a\in V$: la sua classe di equivalenza è formata da tutti quegli elementi di $V$ tali che $  a-  v\in W$, cioè
\begin{equation*}
	[  a]=\{  x\in V\colon  x-  a=  w\in W\}=\{  w+  a\colon  w\in W\}.
\end{equation*}
Questa classe si indica anche come $W+  a$, che si può leggere come l'insieme degli elementi che sono somma di un elemento di $W$ e di $  a$ (e che danno un elemento di $V$): essa si chiama anche \emph{laterale} destro\footnote{In questo caso il laterale è detto \emph{destro} perché il rappresentante $  a$ si trova alla destra dell'operazione. Ovviamente esistono anche laterali sinistri, che in questo caso sarebbero della forma $  a+W$. Trovandoci in spazi vettoriali, però, la somma è commutativa quindi non ha senso distinguere i due casi.} di $W$ in $V$, con rappresentante $  a$.
L'insieme di queste classi di equivalenza distinte, come appena descritte, di $W$ in $V$ si indica con $V\quot W$ e si chiama \emph{spazio quoziente}.

Nello spazio quoziente possiamo definire la somma, che indicheremo con il simbolo $\oplus$, di due laterali come
\begin{equation*}
	(W+  a)\oplus(W+  b)=W+(  a+  b).
\end{equation*}
Se anche fosse $[  a]\equiv[  a']$ e $[  b]\equiv[  b']$, non è detto a priori che si abbia $  a+  b=  a'+  b'$ come conseguenza della proprietà transitiva\footnote{Ovviamente con $  a\neq  a'$ e $  b\neq  b'$, altrimenti sarebbe ovvia: è sufficiente a questo scopo che $  a\sim  a'$, e lo stesso per $  b$.}.
Perché ciò accada serve un'ulteriore condizione, che la relazione sia una \emph{relazione di congruenza}, ossia che sia compatibile con le operazioni in $V$.
Se da $  x\sim  x'$ e $  y\sim  y'$ seguono le relazioni $  x+  y\sim   x'+  y'$ e $\lambda  x\sim\lambda  x'$, allora la relazione $\sim$ è una relazione di congruenza nello spazio vettoriale: mostriamo che la relazione da noi usata per definire lo spazio quoziente lo è.
% Magari aggiungere un enunciato!
\begin{proof}
	Se $  x\sim  x'$, allora $  x-  x'=  w_1\in W$, e analogamente $  y-  y'=  w_2\in W$.
	Sommandoli, si ottiene $  w_1+  w_2=  x-  x'+  y-  y'=(  x+  y)-(  x'+  y')$ che essendo $W$ un sottospazio vettoriale vi appartengono: allora $  x+  y\sim  x'+  y'$.
	Anche per il prodotto con uno scalare $\lambda$, si ottiene analogamente che se $  x-  x'=  w\in W$, allora sempre poiché $W$ è un sottospazio vettoriale si ha che $\lambda  x-\lambda  x'=\lambda  w$ che appartiene a $W$.
\end{proof}
Come per $\oplus$, definiamo anche l'operazione analoga al prodotto scalare, che indichiamo con $\odot$:
\begin{equation*}
	\lambda\odot(W+  a)=W+\lambda  a.
\end{equation*}

La terna $(V\quot W,\oplus,\odot)$ è uno spazio vettoriale su $K$, dove $V$ è a sua volta uno spazio vettoriale sul medesimo campo.
Infatti, oltre alle proprietà appena dimostrate, esiste l'elemento neutro rispetto a $\oplus$, che è $W+0_V$ (si indica anche solamente con $W$), e l'opposto, che è $-(W+  a)=W+(-  a)$.

\section{Algebre}
\begin{definizione} \label{d:algebra}
	Si definisce \emph{algebra} sul campo $K$ uno spazio vettoriale $A$ sul campo $K$ munito di un'ulteriore operazione interna di prodotto, che sia associativo e distributivo rispetto alla somma, ossia tale per cui
	\begin{itemize}
		\item $\forall v,w,z\in A$ si ha: $(v w) z = v (w z)$,
		\item $\forall v,w,z\in A$ e $\forall \lambda\in K$ si ha $(\lambda v + w)z = \lambda v z + w z$.
	\end{itemize}
\end{definizione}
Un'algebra A si dice  \emph{commutativa} se il prodotto è commutativo, si dice \emph{con unità} se $\exists I_a\colon \forall a\in A \text{ si ha } a I_a = I_a a = a$.
